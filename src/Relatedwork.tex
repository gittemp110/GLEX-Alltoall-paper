\section{Related Work}

From an algorithm perspective: Bruck algorithm \cite{bruck1997efficient} is commonly used for small message all-to-all. For mid size messages, isend-irecv algorithm is used. For large messages, linear shift exchange \cite{ranka1994static}, pairwise exchange\cite{thakur2005optimization}.

When considering the multi-core processors:  
Cache-oblivious MPI all-to-all (SH-NUMA-CO) based on morton order is proposed to minimize the cache miss rate \cite{li2017cache}.
For Infiniband and multi-core systems, a non-leader all-to-all collective (SA-orig) which based on shared memory aggregation techniques is proposed in \cite{kumar2008scaling}.
For multi-rail QsNet SMP clusters, a shared memory and RDMA based all-to-all collectives (elan\_alltoall) is proposed in \cite{qian2008efficient}.
For Intel Many Integrated Core (MIC) architecture,  the re-routing scheme based all-to-all collective (PAIRWISE-SLR/BRUCK-SLR) is proposed in \cite{venkatesh2014high}. 
These works are direct related to our work. 
Table \ref{relatedworkcompare} shows the overall design-space comparation between these methods and our method.

%我们用一个表格来展示这些方法和我们的方法的区别。

\begin{table*}[t]
    \caption{Overall design-space for all-to-all collective on mulit-core processers}\label{relatedworkcompare}
    \begin{tabular}{|l|c|c|c|c|c|}
        \hline
        Approach/Metric                  & \multicolumn{1}{l|}{SH-NUMA-CO} & \multicolumn{1}{l|}{SA-orig} & \multicolumn{1}{l|}{elan\_alltoall} & \multicolumn{1}{l|}{\begin{tabular}[c]{@{}l@{}}PAIRWISE-SLR\\ and BRUCK-SLR\end{tabular}} & \multicolumn{1}{l|}{\begin{tabular}[c]{@{}l@{}}GLEX-Alltoall\\ (ours)\end{tabular}} \\ \hline
        inter-node implementation        & MVAPICH2                        & MVAPICH                      & RDMA                                & MPI-P2P                                                                                   & RDMA                                                                                             \\ \hline
        intra-node implementation        & Shared Heap                     & Shared Memory                & Shared Memory                       & MPI-P2P                                                                                   & Shared Heap                                                                                             \\ \hline
        Leader-based aggregation support & \ding{52}                        &\ding{52}                     &\ding{52}                            &\ding{52}                                                                                   &\ding{52}                                                                                                  \\ \hline
        Multi-leader support             &\ding{54}                        &\ding{52}                     &\ding{54}                            &\ding{54}                                                                                  &\ding{52}                                                                                                  \\ \hline
        NUMA-aware                       &\ding{52}                        &\ding{54}                     &\ding{54}                            &\ding{54}                                                                                  &\ding{52}                                                                                                  \\ \hline
        Multiple network endpoints       &\ding{54}                        &\ding{52}                     &\ding{52}                            &\ding{54}                                                                                  &\ding{52}                                                                                                  \\ \hline
        Overlapping Inter/intra-node     &\ding{52}                        &\ding{54}                     &\ding{54}                            &\ding{52}                                                                                  &\ding{52}                                                                                                  \\ \hline
        \end{tabular}
\end{table*}

Besides, to improve the intra-node communication, several kernel-assistant techniques have been proposed for multi-core processor ( LIMIC \cite{jin2007lightweight}, KNEM \cite{goglin2013knem}, XPMEM).
Compared to shared memory, these method works well for large message intra-node communication.
Because they avoid one memory copy overhead.
Shared heap \cite{li2014improved} has the same performance with these kernel-assistant techniques.
Shared heap do not need extra kernel module.

When considering the network topology: A bandwidth-optimal all-to-all exchange is proposed for fat-tree network \cite{prisacari2013bandwidth}. 
For torus network, a large scale all-to-all is proposed for Blue Gene/L Supercomputer \cite{kumar2008optimization}.  
A optimal schedule for all-to-all personalized communication is proposed for multiprocessor systems \cite{saha2019optimal}. 
For Infiniband clusters, their is a topology aware all-to-all scheduler which lower the contention by adding extra communication steps \cite{subramoni2014designing}.


