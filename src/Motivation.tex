\section {Motivation}
Node-aware all-to-all is a traditional method for optimize small message all-to-all.
It include four steps: intra-node gather, local transpose, inter-node all-to-all, intra-node scatter.
With the increase in the number of processor cores, multiple CPU cores, NUMA, network endpoints and inter-/intra-node overlapping brings huge parallelism.
However, we notice that all traditional method do not take advantage of this parallelism optimize the four steps.
For a 32/64 cores CPU, each inter-node message in node-aware method will be 1024/4096 times larger than direct method.
If there is one double data between each pair of processes. There are 1024/4096 double data between each pair of nodes.
As the single-core memory access bandwidth may not faster than the network bandwidth.
Using a single core to gather/scatter/transpose data and initialize communication request may causing communication hotspot happend on a CPU core.

Multiple CPU core can gather/scatter simutaneous using  MPI-RMA. This get faster message aggregation bandwidth.