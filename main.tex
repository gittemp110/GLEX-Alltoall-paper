%\documentclass[review]{elsarticle}
\documentclass[5p,times]{elsarticle}
\usepackage{lineno,hyperref}
\usepackage{amssymb}
\usepackage{stfloats}
\modulolinenumbers[5]
\usepackage{lipsum}
\usepackage{pifont}

\journal{Journal of \LaTeX\ Templates}

%%%%%%%%%%%%%%%%%%%%%%%
%% Elsevier bibliography styles
%%%%%%%%%%%%%%%%%%%%%%%
%% To change the style, put a % in front of the second line of the current style and
%% remove the % from the second line of the style you would like to use.
%%%%%%%%%%%%%%%%%%%%%%%

%% Numbered
%\bibliographystyle{model1-num-names}

%% Numbered without titles
%\bibliographystyle{model1a-num-names}

%% Harvard
%\bibliographystyle{model2-names.bst}\biboptions{authoryear}

%% Vancouver numbered
%\usepackage{numcompress}\bibliographystyle{model3-num-names}

%% Vancouver name/year
%\usepackage{numcompress}\bibliographystyle{model4-names}\biboptions{authoryear}

%% APA style
%\bibliographystyle{model5-names}\biboptions{authoryear}

%% AMA style
%\usepackage{numcompress}\bibliographystyle{model6-num-names}

%% `Elsevier LaTeX' style
\bibliographystyle{elsarticle-num}
%%%%%%%%%%%%%%%%%%%%%%%

\begin{document}

\begin{frontmatter}

\title{GLEX-Alltoall: gather-scatter-based multi-leader All-to-all Communication on Multi-core Supercomputer}

%% Group authors per affiliation:
\author{Jintao Peng\fnref{pjtnote}}
\author{Jie Liu\fnref{ljnote}\corref{mycorrespondingauthor}}
\author{Min Xie\fnref{xmnote}}

\address{Changsha, China}
\fntext[pjtnote]{JintaoPengCS@gmail.com}
\cortext[mycorrespondingauthor]{Corresponding author}
\fntext[ljnote]{liujie@nudt.edu.cn}
\fntext[xmnote]{xiemin@nudt.edu.cn}



\begin{abstract}
		All-to-all communication is commonly used in parallel applications like FFT. 
		In mordern supercomputers, there are multiple cores, NUMAs and network endpoints. 
		These features bring much parallelism. 
		However, there is no method which makes uses the parallelism to improve the all-to-all communication. 
		In this paper, we introduce an optimized NUMA-aware multi-leader all-to-all library which explore the parallelism on network, CPU cores and overlap the intra- and inter-node communication. 
		The results show that, compared to MPI, our library achieves up to 20x speedup. 
		For application, our method achieves up to 1.75x speedup on peak performance for 16384 cores.
\end{abstract}

\begin{keyword}
 Collective Communication \sep Multi-core processor \sep MPI all-to-all \sep RDMA \sep Shared Heap
\MSC[2010] 00-01\sep  99-00
\end{keyword}

\end{frontmatter}

\linenumbers

\section{Introduction}
%对于很多并行应用而言,全局的通信往往成为限制其扩展性的关键点。
%聚合通信作为普遍使用的全局通信,在各种应用种广泛使用。
%all-to-all通信作为聚合通信之一,在FFT和图计算应用中有重要应用。
%然而all-to-all通信在任意两个进程之间都需要发消息。每次并行规模翻倍,all-to-all通信量翻4倍。
%另一方面,网络吞吐率和节点数量的增加则是线性增长。
%这给大规模all-to-all通信带来巨大的挑战。

%典型的方法是在节点内进行聚合消息然后在节点间传输数据。
%简单的说就是将MPI_alltoall替换为n次gather+节点间alltoall+n次scatter。
%该方法对于小消息非常有效。
%通过这种方法消息数量降低节点内进程数量的平方倍。
%在当前超级计算机结构下,有4种并行性有助于优化该节点感知的all-to-all的通信性能:
%多个网络端口使得网卡可以同时处理多个通信请求。
%多个内存控制器/NUMA/内存通道,带来的访存并行性。
%多核心带来的通信请求构造数据Gather和Scatter的并行性。
%节点内和节点间数据传输的并行性。
%但是据我们所知,没有一种节点感知的all-to-all方法充分发掘了这四种并行性。

%在节点内,优化节点感知的all-to-all通信算法在节点内的典型方法有共享内存。
%该方法能够使得节点内通信绕过网卡进行。
%此外,内核辅助的“零拷贝”技术以及共享堆的也是常见的方法。
%这些方法能够降低一次点内的内存拷贝次数。
%同时也有一些方法考虑到了NUMA/UMA中的cache利用效率的问题。
%但是这些方法并没有考虑到随着超级计算机的发展,节点间通信并不会比节点内通信慢太多。
%现代超级计算机普遍装备着多核心,多numa的处理器。
%典型的方法通常没有将这其中的并行性发掘出来。

%本文提出了一种多leader的节点感知all-to-all通信库。
%利用多个核心进行本地转置,发掘多核并行。
%使用多核CPU核心同时进行Gather和Scatter操作,发掘多NUMA并行访存。
%利用多个网络端口同时进行消息传输,发掘网络并行行。
%将节点内和节点间数据传输重叠起来,提升大消息通信性能。
%该通信库在节点内使用共享堆作为Gather和Scatter操作的基础。
%在节点间使用RDMA进行网络通信。
%实验表明,对于16384个进程,我们的通信库对比MPI可以实现最大10X的性能提升。从小消息到大消息平均可提升4.16倍。

Many parallel applications may suffer from global communication.
Especially for communication-intensive applications, their time-to-solution and scalablily may be affected by global communication. 
Message Passing Interface (MPI) provides a set of commonly used collective communication.
MPI\_Alltoall is one of the collective communication where each process will send a different message to all processes.
It is broadly used in some parallel applications like Fast Fourier Transform (FFT) \cite{mehta2021parallel} and graph computing.
However, each time we double the processes, the all-to-all communication workload is quadrupled.
On mordern supercomputers, network throughput has a linear relationship with the number of nodes.
This brings great challenges to large-scale all-to-all communications.

For multiple-core processes, a effective way is node-aware all-to-all method.
It's replace a N nodes global all-to-all into N-1 times intra-node gather + inter-node all-to-all  + N-1 times intra-node gather.
This method is very effective for small messages.
Because, compared to original method, a node-aware all-to-all reduce the number of messages by $M^N$ times (M is number of processers in each nodes).
The size of the message is increased by $M^N$ times, which makes effective use of the network bandwidth.
In the current supercomputer, a node has multiple CPU cores, NUMA and network endpoints.
This architecture brings 4 kinds of parallism to optimize a node-aware all-to-all method:
\begin{enumerate}[(1)]
\item Multiple network endpoints can simultaneously process multiple communication requests.
\item Processes in different NUMA can simultaneously access it local memory without contention.
\item Multiple processes can simultaneously gather/scatter data and compose communication requests.
\item Inter-node communication can be overlapped with intra-node communication.
\end{enumerate}
As we known, no methods combine these parallism togather to improve a node-aware all-to-all collective communication. 

% A typical intra-node communication is based on shared memory.
% Processes create a shared memory at initialization stage, the data is fistly copied from senders buffer into the shared memory buffer, then copied out to the receivers buffer.
% Two data copy needed to transer a intra-node message.
% To reduce the memory copy times, kernel-assisted communication, shared heaps, and thread-based MPI are proposed.

In this paper, we proposed a multi-leader node-aware all-to-all method.
It using multiple leaders on different NUMA which open the different network endpoints to gather/scatter data, compose communication requests, and transpose local matrix.
It explore the parallelism existing in mordern multi-core processor with NUMA memory architecture and multi-port network.
For intra-node gather/sactter, we proposed a shared-heap-based remote accessible memory which similar to intra-node MPI RMA.
Inter-node communication is based on Remote Direct Memory Access (RDMA) which provide high throughput and low latency.
The results show that, compared to MPI\_alltoall, our implementation achieves up to 20x speedup and 4x speedup on average. 
\section{Related Work}

From an algorithm perspective: Bruck algorithm \cite{bruck1997efficient} is commonly used for small message all-to-all. For mid size messages, isend-irecv algorithm is used. For large messages, linear shift exchange \cite{ranka1994static}, pairwise exchange\cite{thakur2005optimization}.

When considering the multi-core processors:  
Cache-oblivious MPI all-to-all (SH-NUMA-CO) based on morton order is proposed to minimize the cache miss rate \cite{li2017cache}.
For Infiniband and multi-core systems, a all-to-all collective (SA-orig) which based on shared memory aggregation techniques is proposed in \cite{kumar2008scaling}.
For multi-rail QsNet SMP clusters, a shared memory and RDMA based all-to-all collectives (elan\_alltoall) is proposed in \cite{qian2008efficient}.
For Intel Many Integrated Core (MIC) architecture,  the re-routing scheme based all-to-all collective (PAIRWISE-SLR/BRUCK-SLR) is proposed in \cite{venkatesh2014high}. 
These works are direct related to our work. 
Table 1 shows the overall design-space for all-to-all collective on mulit-core processers.
%我们用一个表格来展示这些方法和我们的方法的区别。

\begin{table*}[t]
    \begin{tabular}{|l|c|c|c|c|c|}
        \hline
        Approach/Metric                  & \multicolumn{1}{l|}{SH-NUMA-CO} & \multicolumn{1}{l|}{SA-orig} & \multicolumn{1}{l|}{elan\_alltoall} & \multicolumn{1}{l|}{\begin{tabular}[c]{@{}l@{}}PAIRWISE-SLR\\ and BRUCK-SLR\end{tabular}} & \multicolumn{1}{l|}{\begin{tabular}[c]{@{}l@{}}GLEX-Alltoall\\ (Proposed Approach)\end{tabular}} \\ \hline
        inter-node implementation        & MVAPICH2                        & MVAPICH                      & RDMA                                & MPI-P2P                                                                                   & RDMA                                                                                             \\ \hline
        intra-node implementation        & Shared Heap                     & Shared Memory                & Shared Memory                       & MPI-P2P                                                                                   & Shared Heap                                                                                      \\ \hline
        Leader-based aggregation support &                                 &                              &                                     &                                                                                           &                                                                                                  \\ \hline
        Multi-leader support             &                                 &                              &                                     &                                                                                           &                                                                                                  \\ \hline
        NUMA-aware                       &                                 &                              &                                     &                                                                                           &                                                                                                  \\ \hline
        Multiple network endpoints       &                                 &                              &                                     &                                                                                           &                                                                                                  \\ \hline
        Overlapping Inter/intra-node     &                                 &                              &                                     &                                                                                           &                                                                                                  \\ \hline
        \end{tabular}
\end{table*}


When considering the network topology: A bandwidth-optimal all-to-all exchange is proposed for fat-tree network \cite{prisacari2013bandwidth}. 
For torus network, a large scale all-to-all is proposed for Blue Gene/L Supercomputer \cite{kumar2008optimization}.  
A optimal schedule for all-to-all personalized communication is proposed for multiprocessor systems \cite{saha2019optimal}. 
For Infiniband clusters, their is a topology aware all-to-all scheduler which lower the contention by adding extra communication steps \cite{subramoni2014designing}.



When considering the network topology: A bandwidth-optimal all-to-all exchange is proposed for fat-tree network \cite{prisacari2013bandwidth}. 
For torus network, a large scale all-to-all is proposed for Blue Gene/L Supercomputer \cite{kumar2008optimization}.  
A optimal schedule for all-to-all personalized communication is proposed for multiprocessor systems \cite{saha2019optimal}. 
For Infiniband clusters, their is a topology aware all-to-all scheduler which lower the contention by adding extra communication steps \cite{subramoni2014designing}.



When considering the network topology: A bandwidth-optimal all-to-all exchange is proposed for fat-tree network \cite{prisacari2013bandwidth}. 
For torus network, a large scale all-to-all is proposed for Blue Gene/L Supercomputer \cite{kumar2008optimization}.  
A optimal schedule for all-to-all personalized communication is proposed for multiprocessor systems \cite{saha2019optimal}. 
For Infiniband clusters, their is a topology aware all-to-all scheduler which lower the contention by adding extra communication steps \cite{subramoni2014designing}.



When considering the network topology: A bandwidth-optimal all-to-all exchange is proposed for fat-tree network \cite{prisacari2013bandwidth}. 
For torus network, a large scale all-to-all is proposed for Blue Gene/L Supercomputer \cite{kumar2008optimization}.  
A optimal schedule for all-to-all personalized communication is proposed for multiprocessor systems \cite{saha2019optimal}. 
For Infiniband clusters, their is a topology aware all-to-all scheduler which lower the contention by adding extra communication steps \cite{subramoni2014designing}.

When considering the network topology: A bandwidth-optimal all-to-all exchange is proposed for fat-tree network \cite{prisacari2013bandwidth}. 
For torus network, a large scale all-to-all is proposed for Blue Gene/L Supercomputer \cite{kumar2008optimization}.  
A optimal schedule for all-to-all personalized communication is proposed for multiprocessor systems \cite{saha2019optimal}. 
For Infiniband clusters, their is a topology aware all-to-all scheduler which lower the contention by adding extra communication steps \cite{subramoni2014designing}.

When considering the network topology: A bandwidth-optimal all-to-all exchange is proposed for fat-tree network \cite{prisacari2013bandwidth}. 
For torus network, a large scale all-to-all is proposed for Blue Gene/L Supercomputer \cite{kumar2008optimization}.  
A optimal schedule for all-to-all personalized communication is proposed for multiprocessor systems \cite{saha2019optimal}. 
For Infiniband clusters, their is a topology aware all-to-all scheduler which lower the contention by adding extra communication steps \cite{subramoni2014designing}.

When considering the network topology: A bandwidth-optimal all-to-all exchange is proposed for fat-tree network \cite{prisacari2013bandwidth}. 
For torus network, a large scale all-to-all is proposed for Blue Gene/L Supercomputer \cite{kumar2008optimization}.  
A optimal schedule for all-to-all personalized communication is proposed for multiprocessor systems \cite{saha2019optimal}. 
For Infiniband clusters, their is a topology aware all-to-all scheduler which lower the contention by adding extra communication steps \cite{subramoni2014designing}.

When considering the network topology: A bandwidth-optimal all-to-all exchange is proposed for fat-tree network \cite{prisacari2013bandwidth}. 
For torus network, a large scale all-to-all is proposed for Blue Gene/L Supercomputer \cite{kumar2008optimization}.  
A optimal schedule for all-to-all personalized communication is proposed for multiprocessor systems \cite{saha2019optimal}. 
For Infiniband clusters, their is a topology aware all-to-all scheduler which lower the contention by adding extra communication steps \cite{subramoni2014designing}.

When considering the network topology: A bandwidth-optimal all-to-all exchange is proposed for fat-tree network \cite{prisacari2013bandwidth}. 
For torus network, a large scale all-to-all is proposed for Blue Gene/L Supercomputer \cite{kumar2008optimization}.  
A optimal schedule for all-to-all personalized communication is proposed for multiprocessor systems \cite{saha2019optimal}. 
For Infiniband clusters, their is a topology aware all-to-all scheduler which lower the contention by adding extra communication steps \cite{subramoni2014designing}.


\section {Experimental Platforms}
The methods in this paper are validated on four different systems:
\begin{enumerate}[(1)]
\item HPC-A: Matrix-2000+ CPU, Tianhe series network.
\item HPC-B: FT-2000+ CPU, Tianhe series network.
\item HPC-C: Intel E5 CPU, Tianhe series network. 
\item HPC-D: E5-2692 v2 * 2, Tianhe series network. 
\end{enumerate}
The Stream bandwidth test results are shown in the Table \ref{Memorybandwidth}.
\begin{table}[]
    \caption{Memory Bandwidth tested by Stream}\label{Memorybandwidth}
\begin{tabular}{|l|c|c|}
\hline
\begin{tabular}[c]{@{}l@{}}Memory Bandwidth\\ tests (MB/s)\end{tabular} & \begin{tabular}[c]{@{}c@{}}Copy\\ (Single-thread)\end{tabular} & \begin{tabular}[c]{@{}c@{}}Copy\\ (Multi-threads)\end{tabular} \\ \hline
HPC-A (32 threads)                                                      &                                                                &                                                                \\ \hline
HPC-B (32 threads)                                                      &                                                                &                                                                \\ \hline
HPC-C (24 threads)                                                      &                                                                &                                                                \\ \hline
HPC-D (24 threads)                                                      & 11280.5                                                        & 46252.3                                                        \\ \hline
\end{tabular}
\end{table}
\section {Motivation}
Node-aware all-to-all is a traditional method for optimize small message all-to-all.
It include four steps: intra-node gather, local transpose, inter-node all-to-all, intra-node scatter.
With the increase in the number of processor cores, multiple CPU cores, NUMA, network endpoints and inter-/intra-node overlapping brings huge parallelism.
However, we notice that all traditional method do not take advantage of this parallelism optimize the four steps.
For a 32/64 cores CPU, each inter-node message in node-aware method will be 1024/4096 times larger than direct method.
If there is one double data between each pair of processes. There are 1024/4096 double data between each pair of nodes.
As the single-core memory access bandwidth may not faster than the network bandwidth.
Using a single core to gather/scatter/transpose data and initialize communication request may causing communication hotspot happend on a CPU core.

\section{Front matter}

The author names and affiliations could be formatted in two ways:
\begin{enumerate}[(1)]
\item Group the authors per affiliation.
\item Use footnotes to indicate the affiliations.
\end{enumerate}
See the front matter of this document for examples. You are recommended to conform your choice to the journal you are submitting to.

\section{Bibliography styles}

There are various bibliography styles available. You can select the style of your choice in the preamble of this document. These styles are Elsevier styles based on standard styles like Harvard and Vancouver. Please use Bib\TeX\ to generate your bibliography and include DOIs whenever available.

Here are two sample references: \cite{Feynman1963118,Dirac1953888}.

\section*{References}

\bibliography{mybibfile}

\end{document}