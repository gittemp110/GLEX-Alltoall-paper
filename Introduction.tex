\section{Introduction}
%对于很多并行应用而言,全局的通信往往成为限制其扩展性的关键点。
%聚合通信作为普遍使用的全局通信,在各种应用种广泛使用。
%all-to-all通信作为聚合通信之一,在FFT和图计算应用中有重要应用。
%然而all-to-all通信在任意两个进程之间都需要发消息。每次并行规模翻倍,all-to-all通信量翻4倍。
%另一方面,网络吞吐率和节点数量的增加则是线性增长。
%这给大规模all-to-all通信带来巨大的挑战。

%典型的方法是在节点内进行聚合消息然后在节点间传输数据。
%简单的说就是将MPI_alltoall替换为n次gather+节点间alltoall+n次scatter。
%通过这种方法消息数量降低节点内进程数量的平方倍。
%在当前超级计算机结构下,有4种并行性有助于优化该节点感知的all-to-all的通信性能:
%多个网络端口使得网卡可以同时处理多个通信请求。
%多个内存控制器/NUMA/内存通道,带来的访存并行性。
%多核心带来的通信请求构造数据Gather和Scatter的并行性。
%节点内和节点间数据传输的并行性。
%但是据我们所知,没有一种节点感知的all-to-all方法充分发掘了这四种并行性。

%在节点内,优化节点感知的all-to-all通信算法在节点内的典型方法有共享内存。
%该方法能够使得节点内通信绕过网卡进行。
%此外,内核辅助的“零拷贝”技术以及共享堆的也是常见的方法。
%这些方法能够降低一次点内的内存拷贝次数。
%同时也有一些方法考虑到了NUMA/UMA中的cache利用效率的问题。
%但是这些方法并没有考虑到随着超级计算机的发展,节点间通信并不会比节点内通信慢太多。
%现代超级计算机普遍装备着多核心,多numa的处理器。
%典型的方法通常没有将这其中的并行性发掘出来。
